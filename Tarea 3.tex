\documentclass[spanish, fleqn, oneside]{article}
\usepackage{babel}
\usepackage[utf8]{inputenc}
\usepackage{amsmath, amsfonts, amsthm}
\usepackage{mathrsfs}
\usepackage{fourier}
\usepackage{dcolumn}
\usepackage[colorlinks, urlcolor=blue]{hyperref}
\usepackage{fancyhdr}
\usepackage{float}
\usepackage{tikz}
\usepackage{wasysym}
\usepackage{multirow}
\usepackage{pgf}
\usepackage{xcolor}

\setlength{\parindent}{0pt}

\newcommand{\tnum}{3}
\title{Tarea \tnum\\
       Arquitectura de Computadores}
\date{05/06/2023}
\begin{document}
\maketitle

\section{Conceptos básicos}
\begin{enumerate}
    \item ¿Qué es un Branch?, ¿Con que instrucción de un lenguaje de alto nivel lo podemos comparar?(7pts)                                         
    
    \item  ¿Cuál es la diferencia entre Big-endian y Little-endian?, de un ejemplo.(7pts)  

    \item ¿Cuáles son todas las banderas que se pueden encontrar en arm, y que significan?(6pts) 

\end{enumerate}

\newpage
\section{Instrucciones a hexadecimal}

Escriba las siguientes instrucciones en hexadecimal (Asúmalas no condicionadas). 

\begin{enumerate}
    \item SUB R6, R1, R2 (7pts)
 
    \item ADD R4, R6, $\#$0x32 (7pts)

    \item LDR R10, [R3, $\#$24] (8pts)

\end{enumerate}

Determinar el Assembly del siguiente código en lenguaje de máquina: E7933702.(8ptos)\\\\
\textit{*Es necesario poner al menos una tabla en cada ejercicio para demostrar el desarrollo.}
\newpage

\section{Assembly}

Realice una estructura en ARM assembly, en la cual se inicie con dos variables R1, R2 (usted tiene que asignarles el valor a estas variables), la operación tiene que ser una división tipo R1/R2, donde el resultado se tiene que guardar en una variable R0, y el resto en una variable que usted escoja (puede ser una de las variables ya definidas). Se recomienda dar una descripción del código, para que este se entienda mejor.(25pts)\\\\
\textit{*Ocupar verbatim para escribir el código}

\newpage
\section{C a Assembly}

Traducir el siguiente código en C a lenguaje Assembly ARM: 

\begin{verbatim}
#include <stdio.h>

int main() {
    int Var_1 = 0, Var_2 = 0, Var_3 = 1, Tribo;
    for (int i = 0; i < 8; i++) {
        Tribo = Var_1 + Var_2 + Var_3;
        Var_1 = Var_2;
        Var_2 = Var_3;
        Var_3 = Tribo;
    }
    return Var_3;
}
\end{verbatim}
Una breve explicación del código: El código es la implementación de la secuencia de Tribonacci, en donde el valor que retornara la función es la sumatoria de los números de Tribonacci hasta el número 8 en este caso.
(25pts)

\newpage
\section{Condiciones de entrega}
\begin{itemize}
    \item Tienen que subir un único PDF generado por el latex entregado, este se debe llamar “T1 Nombre Apellido ROL” (Con código verificador sin guion).
    
    Ejemplo: ’T1 Juan Doran 2018730501’
    
    \item 
    Todo el desarrollo de este trabajo debe realizarse con las herramientas proporcionadas por Latex y las mencionadas en las preguntas, cualquier imagen
    sacada por camara no se considerara a la hora de revisar.

    \item La entrega en aula se cerrara a las 23:59 del dia Viernes 16 de Junio.
\end{itemize}

\end{document}
